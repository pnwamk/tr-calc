\documentclass{article}
\usepackage{xspace}
\usepackage{amssymb}
\usepackage{amsmath}
\usepackage{amsfonts}
\usepackage{amsthm}
%\usepackage{mathabx}
\usepackage{mathpartir}
\usepackage[margin=0.5in]{geometry}

\usepackage{bbm}
\makeatletter
\newcommand{\ctx}{\ensuremath{\mathsf{ctx}}}
\newcommand{\define}[1]{\textbf{#1}}
\newcommand{\indexdef}[1]{\index{#1|defstyle}}   
\newcommand{\indexsee}[2]{\index{#1|see{#2}}}    
\newcommand{\intro}{\textsc{intro}}
\newcommand{\jdeq}{\equiv}      
\newcommand{\jdeqtp}[4]{#1 \vdash #2 \jdeq #3 : #4}
\def\lamu#1{{\lambda}\@lamuarg#1:\@endlamuarg\@ifnextchar\bgroup{.\,\lamu}{.\,}}
\def\@lamuarg#1:#2\@endlamuarg{#1}
\newcommand{\oftp}[3]{#1 \vdash #2 : #3}
\newcommand{\Subst}{\mathsf{Subst}}
\newcommand{\tmtp}[2]{#1 \mathord{:} #2}
\def\tprd#1{\@tprd{#1}\@ifnextchar\bgroup{\tprd}{}}
\def\@tprd#1{\mathchoice{{\textstyle\prod_{(#1)}}}{\prod_{(#1)}}{\prod_{(#1)}}{\prod_{(#1)}}}
\newcommand{\UU}{\ensuremath{\mathcal{U}}\xspace}
\newcommand{\Vble}{\mathsf{Vble}}
\newcommand{\Weak}{\mathsf{Wkg}}
\newcommand{\wfctx}[1]{#1\ \ctx}
\makeatother

\newcommand{\Ttype}{\mathbf{T}}
\newcommand{\Ftype}{\mathbf{F}}
\newcommand{\Tval}{\# \textrm{t}}
\newcommand{\Fval}{\# \textrm{f}}
\newcommand{\Tprop}{\mathbb{T}}
\newcommand{\Fprop}{\mathbb{F}}
\newcommand{\Ntype}{\mathbf{N}}
\newcommand{\NOT}{\overline}
\newcommand{\oftP}[2]{#1 \vdash #2}
\newcommand{\listof}{\overrightarrow}
\newcommand{\lvec}{\lfloor}
\newcommand{\rvec}{\rfloor}

\begin{document}
\index{structural!rules|(}%
\index{rule!structural|(}%

\begin{tabular}{r  l  l}

$e  ::= $ & $ x \mid 
	            (e \: e) \mid  
	            \lambda x^{\tau}.e \mid 
	            (\mathbf{if} \: e \: e \: e) \mid
	            c \mid
	            \Tval \mid
	            \Fval \mid
	             (cons \: e \: e) \mid
	             (vec \: \listof{e}) \mid
	            n $ & Expressions \\
$c  ::= $ & $ 
	add1 \mid 
	\: = \: \mid  
	\: \leq \: \mid  
	num? \mid 
	 bool? \mid
	 proc? \mid
	 cons? \mid
	 vec? \mid
	 car \mid
	 cdr \mid
	 len \mid
     ref[n] $ & Primitive Operations \\
$ pe ::= $ & $ 
	\mathbf{car} \mid 
	\mathbf{cdr} \mid
	\mathbf{ref}_n \mid
	\mathbf{len} $ & Path Elements \\
$ \pi ::= $ & $ 
	\listof{pe} $ & Paths \\
$ o ::= $ & $ 
	\emptyset \mid 
    \pi (x) $ & Objects \\
$ \phi ::= $ & $ 
	a_0 o_0 + ... + a_n o_n \leq a_{n+1} $ & Linear Inequalities \\
$\sigma , \tau  ::= $ & $ 
	\top \mid 
	\Ntype \mid
	\{x : \Ntype \: | \: \listof{\phi} \} \mid
	\Ttype \mid
	\Ftype  \mid
	(\bigcup \listof{\tau}) \mid
	\langle \tau , \tau \rangle \mid
	\lvec \listof{\tau} \rvec \mid
	x : \sigma \xrightarrow[o]{\psi | \psi} \tau $ & Types \\
$\psi ::= $ & $ 
	\tau_{\pi(x)} \mid 
	\NOT{\tau}_{\pi(x)} \mid  
	\psi \supset \psi \mid 
	\psi \wedge \psi \mid 
	\psi \vee \psi \mid
	\Tprop \mid
	\Fprop  $ & Propositions \\
$ \Gamma ::= $ & $ 
	(\listof{\psi}, \listof{\phi}) $ & Environments \\	
\end{tabular}

% Typing Rules
\begin{mathpar}
  \inferrule[T-Num]
  {}{\oftp{\Gamma}{n}{\{x : \Ntype \: | \: x = n\}   \: ; \: \Tprop \mid \Fprop \: ; \: \emptyset}}

\and
  \inferrule[T-Const]
  {}{\oftp{\Gamma}{c}{\delta_{\tau}(c) \: ; \: \Tprop \mid \Fprop \: ; \: \emptyset}}

\and
  \inferrule[T-True]
  {}{\oftp{\Gamma}{\Tval}{\Ttype \: ; \: \Tprop \mid \Fprop \: ; \: \emptyset}}

\and
  \inferrule[T-False]
  {}{\oftp{\Gamma}{\Fval}{\Ftype \: ; \: \Fprop \mid \Tprop \: ; \: \emptyset}}

\and
  \inferrule[T-Var]
  {\oftP{\Gamma}{\tau_{x}}}
  {\oftp{\Gamma}{x}{\tau \: ; \: \NOT{\Ftype}_x \mid \Ftype_x \: ; \: x}}

\and
  \inferrule[T-Abs]
  {\oftp{\Gamma, \sigma_x}{e}{\tau \: ; \: \psi_+ \mid \psi_- \: ; \: o}}
  {\oftp{\Gamma}{\lambda x^\sigma . e}{ x:\sigma \xrightarrow[o]{\psi_+ | \psi_-} \tau \: ; \: \Tprop \mid \Fprop \: ; \: \emptyset}}


\and
  \inferrule[T-App]
  {\oftp{\Gamma}{e}{x:\sigma \xrightarrow[o_f]{\psi_{f+} | \psi_{f-}} \tau \: ; \: \psi_+ \mid \psi_- \: ; \: o } \\
    \oftp{\Gamma}{e'}{\sigma \: ; \: \psi'_+ \mid \psi'_- \: ; \: o'}}
  {\oftp{\Gamma} {(e \: e')}{ \tau [o' / x]} \: ; \: \psi_{f+} | \psi_{f-}[o' / x] \: ; \: o_f[o' / x]}


\and
  \inferrule[T-If]
  {\oftp{\Gamma}{e_1}{\tau_1 \: ; \: \psi_{1+} \mid \psi_{1-} \: ; \: o_1} \\
    \oftp{\Gamma}{e_2}{\tau \: ; \: \psi_{2+} \mid \psi_{2-} \: ; \: o} \\
    \oftp{\Gamma}{e_3}{\tau \: ; \: \psi_{3+} \mid \psi_{3-} \: ; \: o}}
  {\oftp{\Gamma}{(\mathbf{if} \: e_1 \: e_2 \: e_3)}{ \tau \: ; \: \psi_{2+} \vee \psi_{3+} \mid \psi_{2-} \vee \psi_{3-} \: ; \: o}}

\and
  \inferrule[T-Let]
  {\oftp{\Gamma}{e_0}{\tau \: ; \: \psi_{0+} \mid \psi_{0-} \: ; \: o_0} \\
    \oftp{\Gamma, \tau_x, \NOT{\Ftype}_x \supset \psi_{0+}, \Ftype_x \supset \psi_{0-}}{e_1}{\sigma \: ; \: \psi_{1+} \mid \psi_{1-} \: ; \: o_1}}
  {\oftp{\Gamma}{(\mathbf{let} \: (x \: e_0) \: e_1)}{ \sigma [o_0 / x] \: ; \: \psi_{1+} \mid \psi_{1-} [o_0 / x] \: ; \: o_1 [o_0 / x]}}


\and
  \inferrule[T-Cons]
  {\oftp{\Gamma}{e_1}{\tau_1 \: ; \: \psi_{1+} \mid \psi_{1-} \: ; \: o_1} \\
    \oftp{\Gamma}{e_2}{\tau_2 \: ; \: \psi_{2+} \mid \psi_{2-} \: ; \: o_2}}
  {\oftp{\Gamma}{(cons \: e_1 \: e_2)}{ \langle \tau_1, \tau_2 \rangle \: ; \: \Tprop \mid \Fprop \: ; \: \emptyset}}

\and
  \inferrule[T-Car]
  {\oftp{\Gamma}{e}{\langle \tau_1, \tau_2 \rangle \: ; \: \psi_{+} \mid \psi_{-} \: ; \: o}}
  {\oftp{\Gamma}{(car \: e)}{ \tau_1 \: ; \: \NOT{\Ftype}_{\mathbf{car}(x)} \mid \Ftype_{\mathbf{car}(x)} [o / x] \: ; \: \mathbf{car}(x) [o / x]}}

\and
  \inferrule[T-Cdr]
  {\oftp{\Gamma}{e}{\langle \tau_1, \tau_2 \rangle \: ; \: \psi_{+} \mid \psi_{-} \: ; \: o}}
  {\oftp{\Gamma}{(cdr \: e)}{ \tau_2 \: ; \: \NOT{\Ftype}_{\mathbf{cdr}(x)} \mid \Ftype_{\mathbf{cdr}(x)} [o / x] \: ; \: \mathbf{cdr}(x) [o / x]}}

\and
  \inferrule[T-Vec]
  {\oftp{\Gamma}{e_1}{\tau_1 \: ; \: \psi_{1+} \mid \psi_{1-} \: ; \: o_1} \\
    \dots \\
    \oftp{\Gamma}{e_n}{\tau_n \: ; \: \psi_{n+} \mid \psi_{n-} \: ; \: o_n}}
  {\oftp{\Gamma}{(vec \: e_1 \: \dots \: e_n)}{ \lvec \tau_1 \: \dots \: \tau_n \rvec \: ; \: \Tprop \mid \Fprop \: ; \: \emptyset}}

\and
  \inferrule[T-Ref]
  {\oftp{\Gamma}{e}{\lvec \dots \tau_i \dots \rvec \: ; \: \psi_{i+} \mid \psi_{i-} \: ; \: o_i}}
  {\oftp{\Gamma}{(ref[i] \: \: e)}{ \tau_i \: ; \: \NOT{\Ftype}_{\mathbf{ref}[i](x)} \mid \Ftype_{\mathbf{ref}[i](x)} [o / x] \: ; \: \mathbf{ref}[i](x) [o / x]}}


\and
  \inferrule[T-Subsume]
  {\oftp{\Gamma}{e}{\tau \: ; \: \psi_+ \mid \psi_- \: ; \: o} \\
    \oftP{\Gamma, \psi_+}{\psi'_+} \\
    \oftP{\Gamma, \psi_-}{\psi'_-} \\
    \vdash \tau <: \tau' \\
    \vdash o <: o'}
  {\oftp{\Gamma}{e}{ \tau' \: ; \: \psi'_+ \mid \psi'_- \: ; \: o'}}



\end{mathpar}

% x : \sigma \xrightarrow[o]{\psi_+ | \psi_-} \tau



%
\begin{mathpar}
  \inferrule*[right=$\Vble$]
  {\wfctx {(\tmtp{x_1}{A_1}, \ldots, \tmtp{x_n}{A_n})} }
  {\oftp{\tmtp{x_1}{A_1}, \ldots, \tmtp{x_n}{A_n}}{x_i}{A_i}}
\end{mathpar}
%
As with $\ctx$-\textsc{ext}, the hypothesis and conclusion of the rule $\Vble$ are judgments of different forms, only now they are reversed: we start with a well-formed context and derive a typing judgment.

The following important principles, called \define{substitution}
\indexdef{rule!of substitution}%
and
\define{weakening},
\indexdef{rule!of weakening}%
need not be explicitly assumed. Rather, it is possible to
show, by induction on the structure of all possible derivations, that whenever
the hypotheses of these rules are derivable, their conclusion is also
derivable.\footnote{Such rules are called \define{admissible}\indexdef{rule!admissible}\indexsee{admissible!rule}{rule, admissible}.}
For the typing judgments these principles are manifested as
%
\begin{mathpar}
  \inferrule*[right=$\Subst_1$]
  {\oftp\Gamma{a}{A} \\ \oftp{\Gamma,\tmtp xA,\Delta}{b}{B}}
  {\oftp{\Gamma,\Delta[a/x]}{b[a/x]}{B[a/x]}}
\and
  \inferrule*[right=$\Weak_1$]
  {\oftp\Gamma{A}{\UU_i} \\ \oftp{\Gamma,\Delta}{b}{B}}
  {\oftp{\Gamma,\tmtp xA,\Delta}{b}{B}}
\end{mathpar}
and for judgmental equalities they become
\begin{mathpar}
  \inferrule*[right=$\Subst_2$]
  {\oftp\Gamma{a}{A} \\ \jdeqtp{\Gamma,\tmtp xA,\Delta}{b}{c}{B}}
  {\jdeqtp{\Gamma,\Delta[a/x]}{b[a/x]}{c[a/x]}{B[a/x]}}
\and
  \inferrule*[right=$\Weak_2$]
  {\oftp\Gamma{A}{\UU_i} \\ \jdeqtp{\Gamma,\Delta}{b}{c}{B}}
  {\jdeqtp{\Gamma,\tmtp xA,\Delta}{b}{c}{B}}
\end{mathpar}
%
In addition to the judgmental equality rules given for each type former, we also
assume that judgmental equality is an equivalence relation respected by typing.
\begin{mathparpagebreakable}
  \inferrule*{\oftp\Gamma{a}{A}}{\jdeqtp\Gamma{a}{a}{A}}
\and
  \inferrule*{\jdeqtp\Gamma{a}{b}{A}}{\jdeqtp\Gamma{b}{a}{A}}
\and
  \inferrule*{\jdeqtp\Gamma{a}{b}{A} \\ \jdeqtp\Gamma{b}{c}{A}}{\jdeqtp\Gamma{a}{c}{A}}
\and
  \inferrule*{\oftp\Gamma{a}{A} \\ \jdeqtp\Gamma{A}{B}{\UU_i}}{\oftp\Gamma{a}{B}}
\and
  \inferrule*{\jdeqtp\Gamma{a}{b}{A} \\ \jdeqtp\Gamma{A}{B}{\UU_i}}{\jdeqtp\Gamma{a}{b}{B}}
\end{mathparpagebreakable}
%
Additionally, for all the type formers below, we assume rules stating that each constructor preserves definitional equality in each of its arguments; for instance, along with the $\Pi$-\intro\ rule, we assume the rule
\[
  \inferrule*[right=$\Pi$-\intro-eq]
  {\jdeqtp\Gamma{A}{A'}{\UU_i} \\
   \jdeqtp{\Gamma,\tmtp xA}{B}{B'}{\UU_i} \\
   \jdeqtp{\Gamma,\tmtp xA}{b}{b'}{B}}
  {\jdeqtp\Gamma{\lamu{x:A} b}{\lamu{x:A'} b'}{\tprd{x:A} B}}
\]
However, we omit these rules for brevity.

\index{rule!structural|)}%
\index{structural!rules|)}%

\end{document}
