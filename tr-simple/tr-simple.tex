\documentclass{article}
\usepackage{xspace}
\usepackage{amssymb}
\usepackage{amsmath}
\usepackage{amsfonts}
\usepackage{amsthm}
%\usepackage{mathabx}
\usepackage{mathpartir}
\usepackage[margin=0.5in]{geometry}

\usepackage{bbm}
\makeatletter
\newcommand{\ctx}{\ensuremath{\mathsf{ctx}}}
\newcommand{\define}[1]{\textbf{#1}}
\newcommand{\indexdef}[1]{\index{#1|defstyle}}   
\newcommand{\indexsee}[2]{\index{#1|see{#2}}}    
\newcommand{\intro}{\textsc{intro}}
\def\lamu#1{{\lambda}\@lamuarg#1:\@endlamuarg\@ifnextchar\bgroup{.\,\lamu}{.\,}}
\def\@lamuarg#1:#2\@endlamuarg{#1}
\newcommand{\Subst}{\mathsf{Subst}}
\newcommand{\tmtp}[2]{#1 \mathord{:} #2}
\def\tprd#1{\@tprd{#1}\@ifnextchar\bgroup{\tprd}{}}
\def\@tprd#1{\mathchoice{{\textstyle\prod_{(#1)}}}{\prod_{(#1)}}{\prod_{(#1)}}{\prod_{(#1)}}}
\newcommand{\UU}{\ensuremath{\mathcal{U}}\xspace}
\newcommand{\Vble}{\mathsf{Vble}}
\newcommand{\Weak}{\mathsf{Wkg}}
\newcommand{\wfctx}[1]{#1\ \ctx}
\makeatother

\newcommand{\Ttype}{\mathbf{T}}
\newcommand{\Ftype}{\mathbf{F}}
\newcommand{\Tval}{\# \textrm{t}}
\newcommand{\Fval}{\# \textrm{f}}
\newcommand{\Tprop}{\mathbb{T}}
\newcommand{\Fprop}{\mathbb{F}}
\newcommand{\Uprop}{\mathbb{U}}
\newcommand{\Ntype}{\mathbf{N}}
\newcommand{\Btype}{\mathbf{B}}
\newcommand{\NOT}[1]{\overline{#1}}
\newcommand{\TypeOf}[5]{#1 \vdash #2 : #3 \: ; \: #4 \: ; \: #5}
\newcommand{\Proves}[2]{#1 \vdash #2}
\newcommand{\listof}{\overrightarrow}
\newcommand{\lvec}{[[}
\newcommand{\rvec}{]]}
\newcommand{\vectype}[1]{\lvec #1 \rvec}
\newcommand{\funtype}[6]{#1\mathord{:}#2 \xrightarrow[ #5 ]{ #3 \mid #4 } #6 }
\newcommand{\deptype}[3]{\{ #1 : #2 \: | \: #3 \}}
\newcommand{\pairtype}[2]{\langle #1 , #2 \rangle}
\newcommand{\U}{\bigcup}
\newcommand{\NullO}{\emptyset}
\newcommand{\update}[3]{\textrm{update}(#1, #2, #3)}
\newcommand{\carpe}{\mathbf{car}}
\newcommand{\cdrpe}{\mathbf{cdr}}
% \newcommand{\refpe}[1]{\mathbf{ref}_{#1}}
\newcommand{\lenpe}{\mathbf{len}}
\newcommand{\lexp}{\mathfrak{L}}

\begin{document}
\index{structural!rules|(}%
\index{rule!structural|(}%

\begin{figure}
\begin{tabular}{r  l  l}

$e  ::= $ & 
  $ x \mid 
  (e \: e) \mid  
  \lambda x^{\tau}.e \mid 
  (\mathbf{if} \: e \: e \: e) \mid
  c \mid
  \Tval \mid
  \Fval \mid
  n $ & Expressions \\
$c  ::= $ & $ 
  add1 \mid
  zero? \mid 
  num? \mid 
  bool? \mid
  proc? $ & Primitive Operations \\
$ o ::= $ & 
  $ x $ & Objects \\
$\sigma , \tau  ::= $ & $ 
	\top \mid
    \Ntype \mid 
	\Ttype \mid
	\Ftype  \mid
	(\U \listof{\tau}) \mid
	\funtype{x}{\sigma}{\psi}{\psi}{o}{\tau} $ & Types \\
$\psi ::= $ & $ 
	\tau_{\pi(x)} \mid 
	\NOT{\tau}_{\pi(x)} \mid 
	\psi \wedge \psi \mid 
	\psi \vee \psi \mid 
	\Tprop \mid 
	\Fprop $ & Propositions \\
$ \Gamma ::= $ & $ 
	\listof{\psi} $ & Environments \\	
\end{tabular}
\\
\\
$ \bot $ is defined as $(\U)$, $\Btype$ is defined as $(\U \: \Ttype \: \: \Ftype)$.
\caption{Syntax of Types, Propositions, Terms, etc...}
\end{figure}

\end{document}
